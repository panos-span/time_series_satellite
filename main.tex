\documentclass[a4paper,12pt]{article}
\usepackage[utf8]{inputenc}
\usepackage[LGR,T1]{fontenc}
\usepackage{alphabeta}
\usepackage{amsmath}
\usepackage{float}
\usepackage{multirow}
\usepackage{amsfonts}
\usepackage{amssymb}
\usepackage{graphicx}
\usepackage{listings}
\usepackage{booktabs}  % For better looking tables
\usepackage{hyperref}
\usepackage{xcolor}
\usepackage{breakurl}
\usepackage[numbers]{natbib}
\usepackage{geometry}
\usepackage{textcomp}
\usepackage{fancyhdr}

\lstset{
  basicstyle=\ttfamily\footnotesize,
  keywordstyle=\color{blue},
  commentstyle=\color{gray},
  stringstyle=\color{red},
  showstringspaces=false,
  breaklines=true,
  frame=single,
  language=R,
  extendedchars=true,
  literate={β}{{\beta}}1
}

% Code listing style
\lstset{
    language=Python,
    basicstyle=\ttfamily\footnotesize,
    keywordstyle=\color{blue},
    commentstyle=\color{green!60!black},
    stringstyle=\color{red},
    numberstyle=\tiny\color{gray},
    numbers=left,
    numbersep=5pt,
    breaklines=true,
    frame=single,
    captionpos=b,
    tabsize=2,
    showstringspaces=false,
    backgroundcolor=\color{gray!10}
}

\title{Earth Observation Big Data and Analytics\\ 3rd Exercise \\ Fall Semester 2024-2025 \\ Ε.ΔE.ΜΜ}
\author{Παναγιώτης-Αλέξιος Σπανάκης (ΑΜ: 03400274)}
\date{\today}

\begin{document}

\maketitle

\tableofcontents
\newpage

\section{Introduction}

\subsection{Background and Motivation}

The increasing availability of Earth observation data and advances in
geospatial big data processing have revolutionized environmental monitoring and
spatial analysis capabilities. This study focuses on the Kastoria region in
northern Greece, an area characterized by the distinctive Lake Orestiada and
diverse land cover patterns that provide an excellent case study for
multi-temporal analysis.

The integration of satellite remote sensing data with meteorological
observations and administrative vector datasets enables comprehensive
environmental assessment and monitoring. This project demonstrates a complete
geospatial analysis workflow, from data acquisition and preprocessing to
interactive visualization and web deployment.

\subsection{Objectives}

The primary objectives of this analysis are:

\begin{enumerate}
    \item \textbf{Data Integration:} Acquire and integrate multi-source geospatial datasets including Sentinel-2 time series, meteorological data from NASA POWER API, and vector data from Greek national geodata portal

    \item \textbf{Raster Analysis:} Implement spectral indices calculation (NDVI, NDWI, BSI) with proper polygon-based masking and temporal analysis

    \item \textbf{Time Series Visualization:} Create comprehensive static maps and temporal visualizations of both satellite-derived indices and meteorological variables

    \item \textbf{Interactive Web Applications:} Develop professional web-based mapping applications integrating all data sources with OGC web services
\end{enumerate}

\subsection{Study Area}

The Kastoria study area is located in northwestern Greece (approximately
40.52°N, 21.27°E), encompassing the historic city of Kastoria and the
surrounding landscape including Lake Orestiada. The region exhibits diverse
land cover types including urban areas, agricultural land, forests, and the
prominent freshwater lake system.

The study area was precisely defined using a polygon geometry created through
geojson.io, covering approximately 89 km² and encompassing the main urban
center, lake system, and surrounding rural landscape. This area provides an
excellent test case for demonstrating polygon-based analysis techniques and
multi-temporal monitoring capabilities.

\section{Data Sources and Methodology}

\subsection{Data Sources}

\subsubsection{Sentinel-2 Time Series Data}

The primary satellite dataset consists of multi-temporal Sentinel-2 imagery
covering the Kastoria region:

\begin{itemize}
    \item \textbf{Source:} Greek Research \& Technology Network (GRNET) - Pithos Cloud Storage
    \item \textbf{Format:} Kastoria.tif (240 bands representing 24 timesteps × 10 spectral bands)
    \item \textbf{Spatial Resolution:} ~10 meters
    \item \textbf{Spectral Bands:} Blue, Green, Red, Red Edge (multiple), NIR, SWIR
    \item \textbf{Reference Data:} Kast\_RefData\_26Classes.tif (26-class land cover classification)
\end{itemize}

\subsubsection{Meteorological Data}

Meteorological time series data was acquired from NASA's POWER (Prediction of
Worldwide Energy Resources) API:

\begin{itemize}
    \item \textbf{Source:} NASA POWER API
    \item \textbf{Variables:} Temperature (T2M), Precipitation (PRECTOTCORR), Relative Humidity (RH2M), Wind Speed (WS2M)
    \item \textbf{Temporal Resolution:} Daily
    \item \textbf{Period:} 2018-2024
    \item \textbf{Location:} Study area centroid (40.513°N, 21.281°E)
\end{itemize}

\subsubsection{Vector Data}

Administrative and infrastructure vector data was obtained from the Greek
National Geodata Portal:

\begin{itemize}
    \item \textbf{Source:} geodata.gov.gr
    \item \textbf{Services:} WFS (Web Feature Service) and WMS (Web Map Service)
    \item \textbf{Datasets:} Administrative boundaries, road networks, settlements
    \item \textbf{Format:} GeoJSON through WFS requests
\end{itemize}

\subsection{Software and Libraries}

The analysis was implemented using Python 3.x with the following key libraries:

\begin{itemize}
    \item \textbf{Geospatial Processing:} rasterio, geopandas, shapely
    \item \textbf{Data Analysis:} pandas, numpy, scipy
    \item \textbf{Visualization:} matplotlib, seaborn, folium, leafmap
    \item \textbf{Web Services:} requests, OWSLib
\end{itemize}

\subsection{Methodology}

\subsubsection{Study Area Definition and Validation}

The study area polygon was manually digitized using geojson.io to ensure
precise boundary definition that encompasses the Kastoria urban area and Lake
Orestiada. A comprehensive validation system was implemented using 14 known
landmarks:

\begin{lstlisting}[caption=Landmark-based polygon validation system]
kastoria_landmarks = {
    'Kastoria_City_Center': (21.2685, 40.5167),
    'Kastoria_Lake_Center': (21.2750, 40.5150),
    'Byzantine_Museum': (21.2672, 40.5189),
    'Olympic_Stadium': (21.2584, 40.5201),
    'Dragon_Cave': (21.2612, 40.5089),
    # Additional landmarks...
}
\end{lstlisting}

The validation achieved 100\% coverage of key landmarks, confirming the polygon
accurately represents the intended study area.

\textbf{Scientific Importance of Landmark Validation:} The landmark-based
validation approach serves multiple critical purposes: (1) ensures spatial
accuracy of remotely digitized boundaries, (2) provides ground-truth validation
for polygon coverage, (3) confirms inclusion of culturally and ecologically
significant features, and (4) enables reproducible boundary definition for
future studies. This rigorous validation is essential for scientific credibility,
particularly when polygon boundaries significantly influence spectral analysis
results through masking operations.

\subsubsection{Raster Data Processing}

A critical component of this analysis was the proper implementation of
rasterio.mask for polygon-based raster clipping. Initial attempts using
incorrect parameters resulted in 1D flattened arrays due to improper geometry
formatting and missing the \texttt{crop=True} parameter. The issue was
identified when masked arrays returned unexpected dimensions (889,301,) instead
of the expected 2D spatial arrays. This was resolved through proper parameter
specification and geometry preparation:

\begin{lstlisting}[caption=Correct rasterio.mask implementation]
masked_array, masked_transform = mask(
    dataset=src,
    shapes=geometries,      # List of geometries
    crop=True,             # Crop to geometry extent  
    indexes=[band_index],  # 1-based band indexing
    nodata=np.nan,         # Set nodata value
    filled=True,           # Return filled array
    all_touched=False      # Only center pixels
)
\end{lstlisting}

This approach ensures precise spatial analysis limited to pixels within the
study area polygon, providing scientifically accurate results.

\subsubsection{Data Quality and Preprocessing}

Quality assurance measures were implemented throughout the analysis workflow:

\textbf{Raster Data Validation:} Sentinel-2 bands were validated for proper
spatial extent, coordinate reference system consistency, and radiometric
integrity. The 240-band structure was verified to ensure correct temporal
sequencing (24 timesteps × 10 spectral bands).

\textbf{Meteorological Data Quality:} NASA POWER data underwent quality checks
including temporal continuity assessment, outlier detection, and statistical
validation against regional climate normals.

\textbf{Polygon Geometry Validation:} Beyond landmark checking, polygon
topology was validated for self-intersection, proper closure, and coordinate
precision to ensure robust masking operations.

\textbf{Nodata Handling:} Comprehensive nodata value management using np.nan
for raster operations ensures accurate statistical calculations and prevents
artifacts in visualization and analysis results.

\subsubsection{Temporal Analysis Scope and Rationale}

For this analysis, we focused on the first 6 timesteps out of the 24 available
timesteps in the Sentinel-2 dataset. This subset was selected for several
reasons:

\begin{enumerate}
    \item \textbf{Methodological Validation:} Demonstrating the correct
          implementation of polygon-based masking across multiple timesteps
    \item \textbf{Computational Efficiency:} Processing 6 timesteps allows for
          comprehensive analysis while maintaining reasonable computational
          requirements
    \item \textbf{Seasonal Representation:} The selected timesteps span
          different seasonal conditions, enabling detection of temporal patterns
    \item \textbf{Quality Assessment:} Validating data consistency and
          methodology before scaling to the full temporal series
\end{enumerate}

The methodology established can be directly applied to analyze all 24 timesteps
for comprehensive long-term monitoring studies.

\subsubsection{Spectral Indices Calculation}

Three key spectral indices were calculated for vegetation and land cover
analysis:

\begin{enumerate}
    \item \textbf{NDVI (Normalized Difference Vegetation Index):}
          \begin{equation}
              NDVI = \frac{NIR - Red}{NIR + Red}
          \end{equation}

    \item \textbf{NDWI (Normalized Difference Water Index):}
          \begin{equation}
              NDWI = \frac{Green - NIR}{Green + NIR}
          \end{equation}

    \item \textbf{BSI (Bare Soil Index):}
          \begin{equation}
              BSI = \frac{(Red + SWIR1) - (NIR + Blue)}{(Red + SWIR1) + (NIR + Blue)}
          \end{equation}
\end{enumerate}

\section{Results}

\subsection{Objective 1: Data Integration and Validation}

\subsubsection{Study Area Characterization}

The defined study area polygon encompasses 89.05 km² and successfully covers all
major landmarks within the Kastoria region. The polygon validation system
confirmed 100\% coverage of critical infrastructure including the city center,
lake system, and key cultural sites.

\begin{table}[H]
    \centering
    \caption{Study Area Characteristics}
    \begin{tabular}{@{}ll@{}}
        \toprule
        Parameter            & Value                \\
        \midrule
        Total Area           & 89.05 km²            \\
        Centroid Coordinates & 40.513°N, 21.281°E  \\
        Geometry Type        & Polygon (7 vertices) \\
        Landmark Coverage    & 14/14 (100\%)        \\
        CRS                  & EPSG:4326 (WGS84)    \\
        \bottomrule
    \end{tabular}
\end{table}

\subsubsection{Data Loading and Integration}

All data sources were successfully integrated:

\begin{itemize}
    \item \textbf{Sentinel-2 Data:} 240 bands (24 timesteps × 10 spectral bands), analyzed first 6 timesteps
    \item \textbf{Meteorological Data:} 2,191 daily records (2018-2024)
    \item \textbf{Vector Data:} Administrative boundaries, road networks, settlements
\end{itemize}

\begin{figure}[H]
    \centering
    \includegraphics[width=0.8\textwidth]{figures/kastoria_study_area_overview.png}
    \caption{Kastoria study area polygon overview showing the precisely defined boundaries covering 97.2 km². The polygon encompasses Kastoria city center, Lake Orestiada, and surrounding landscape. Center coordinates: 40.513°N, 21.281°E. The blue star indicates the meteorological station location for NASA POWER data collection.}
    \label{fig:study_area}
\end{figure}

\subsubsection{RGB Band Visualization}

Individual spectral bands and their composite visualization demonstrate the quality of the Sentinel-2 data:

\begin{figure}[H]
    \centering
    \includegraphics[width=\textwidth]{figures/kastoria_rgb_bands.png}
    \caption{RGB band visualization for timestep 0 showing individual spectral bands and composite. (a) Red band showing vegetation and urban areas; (b) Green band highlighting vegetation patterns; (c) Blue band indicating atmospheric and water features; (d) RGB composite providing natural color representation of the Kastoria study area with Lake Orestiada clearly visible as the dark water body.}
    \label{fig:rgb_bands}
\end{figure}

\subsection{Objective 2: Raster Analysis and Spectral Indices}

\subsubsection{Sentinel-2 Time Series Analysis}

The Sentinel-2 time series analysis reveals clear temporal patterns across the
study period, with Lake Kastoria consistently identifiable in all timesteps
while surrounding vegetation shows distinct seasonal variations from winter
(timesteps 0-6) to summer (timesteps 18-23).

\begin{figure}[H]
    \centering
    \includegraphics[width=\textwidth]{figures/kastoria_sentinel_timeseries.png}
    \caption{Sentinel-2 time series analysis showing temporal variations across five representative timesteps and reference classification. Bands 0, 60, 120, 180, and 239 represent timesteps 0, 6, 12, 18, and 23 respectively, demonstrating seasonal changes from winter to summer. Lake Kastoria remains consistently identifiable (dark areas) while surrounding vegetation shows distinct seasonal patterns. Reference classification shows 26 land cover classes.}
    \label{fig:sentinel_timeseries}
\end{figure}

\subsubsection{Spectral Indices Analysis}

The spectral indices analysis revealed distinct spatial patterns corresponding
to different land cover types:

\begin{figure}[H]
    \centering
    \includegraphics[width=\textwidth]{figures/kastoria_spectral_indices.png}
    \caption{Spectral indices calculated for timestep 0 using proper polygon masking. (a) NDVI shows vegetation distribution with mean -0.290 ± 0.517, with Lake Kastoria appearing as low values (red/dark red); (b) NDWI clearly highlights Lake Kastoria as high values (dark blue) with mean 0.288 ± 0.515; (c) BSI indicates bare soil and urban areas with mean 0.143 ± 0.318. Valid pixels: 889,301 out of 1,269,168 total, demonstrating precise polygon-based analysis.}
    \label{fig:spectral_indices}
\end{figure}

\subsubsection{Spectral Indices Interpretation and Lake Kastoria Influence}

The spectral indices analysis reveals the dominant influence of Lake Kastoria
within the study area, providing valuable insights into land-water dynamics:

\textbf{NDVI Analysis:} The negative mean NDVI (-0.262) is scientifically
accurate and reflects the substantial water coverage. Water bodies absorb
strongly in the near-infrared spectrum while reflecting in the visible red,
resulting in negative NDVI values. The high standard deviation (±0.517)
indicates the heterogeneous landscape mixing water, vegetation, and urban areas.

\textbf{NDWI Analysis:} The positive NDWI values (mean: 0.267) confirm
successful water body delineation. Lake Kastoria appears as high NDWI values
(dark blue in visualization), demonstrating the index's effectiveness for water
mapping in this Mediterranean environment.

\textbf{BSI Analysis:} The moderate BSI values (mean: 0.142) indicate
relatively low bare soil exposure, suggesting either vegetation cover or water
dominance. The decreasing trend suggests potential vegetation recovery or
seasonal greening.

This multi-index approach provides complementary information: NDVI for overall
vegetation assessment (despite water dominance), NDWI for precise water
delineation, and BSI for soil exposure monitoring.

\subsubsection{Time Series Analysis}

The temporal analysis of spectral indices shows clear seasonal patterns and
trends:

\begin{table}[H]
    \centering
    \caption{Spectral Indices Statistics Summary (First 6 Timesteps)}
    \begin{tabular}{@{}lcccc@{}}
        \toprule
        Index & Mean ± Std     & Min    & Max   & Trend      \\
        \midrule
        NDVI  & -0.262 ± 0.517 & -0.400 & 0.000 & Increasing \\
        NDWI  & 0.267 ± 0.515  & -0.000 & 0.400 & Decreasing \\
        BSI   & 0.142 ± 0.318  & 0.020  & 0.200 & Decreasing \\
        \bottomrule
    \end{tabular}
\end{table}

\begin{figure}[H]
    \centering
    \includegraphics[width=\textwidth]{figures/kastoria_time_series.png}
    \caption{Time series evolution of spectral indices over 6 timesteps using correct polygon masking. (a) NDVI shows overall mean -0.262 with increasing trend; (b) NDWI indicates water content variations with overall mean 0.267 and decreasing trend; (c) BSI reflects bare soil exposure with overall mean 0.142 and decreasing trend. Error bands show spatial variability (±1σ) within the study area.}
    \label{fig:time_series}
\end{figure}

The polygon masking successfully restricted analysis to 889,301 valid pixels
within the study area boundary, representing 70.1\% spatial coverage. This
demonstrates the effectiveness of the rasterio.mask implementation and ensures
scientifically accurate spatial analysis limited to the defined study area.

\subsection{Objective 3: Static Maps and Meteorological Analysis}

\subsubsection{Geographic Context Maps}

Four comprehensive static maps were created to provide complete geographic
context:

\begin{figure}[H]
    \centering
    \includegraphics[width=\textwidth]{figures/kastoria_static_maps.png}
    \caption{Comprehensive static maps showing: (a) Study area with administrative boundaries and meteorological station; (b) Transportation networks with road classifications; (c) Detailed study area view with coordinates annotation; (d) Regional context showing wider geographic area. The study area polygon (red) covers 89.05 km² with the meteorological station marked as a blue star.}
    \label{fig:static_maps}
\end{figure}

\subsubsection{Meteorological Time Series}

The meteorological analysis reveals clear temporal patterns and correlations
with spectral indices:

\begin{table}[H]
    \centering
    \caption{Meteorological Statistics (2018-2024)}
    \begin{tabular}{@{}lcccc@{}}
        \toprule
        Variable               & Mean ± Std   & Min  & Max  & Trend      \\
        \midrule
        Temperature (°C)       & 11.68 ± 8.57 & -8.2 & 29.7 & Increasing \\
        Precipitation (mm/day) & 1.89 ± 4.20  & 0.0  & 52.2 & Stable     \\
        Humidity (\%)          & 71.01 ± 16.97& 24.8 & 99.5 & Stable     \\
        Wind Speed (m/s)       & 1.81 ± 0.80  & 0.45 & 5.94 & Decreasing \\
        \bottomrule
    \end{tabular}
\end{table}

\begin{figure}[H]
    \centering
    \includegraphics[width=\textwidth]{figures/kastoria_meteorological_timeseries.png}
    \caption{Meteorological time series for Kastoria study area (2018-2024) from NASA POWER API. (a) Temperature shows clear seasonal cycles with mean 11.68 ± 8.57°C and slight warming trend; (b) Precipitation exhibits high variability with mean 1.89 ± 4.20 mm/day; (c) Relative humidity remains stable with mean 71.01 ± 16.97\%; (d) Wind speed shows mean 1.81 ± 0.80 m/s with gradual decline. Trend lines indicate long-term patterns.}
    \label{fig:met_timeseries}
\end{figure}

\subsubsection{Meteorological-Spectral Correlations}

The integration of meteorological and spectral data reveals important
environmental correlations:

\textbf{Temperature-Vegetation Relationship:} The seasonal temperature patterns
(Spring: 9.4°C to Summer: 22.0°C) correlate with the NDVI increasing trend
observed in the time series, suggesting vegetation response to warming
temperatures during the growing season.

\textbf{Precipitation Influence:} The high precipitation variability (1.89 ±
4.20 mm/day) with winter maximum aligns with NDWI patterns, indicating seasonal
water availability changes that affect both lake levels and surrounding
vegetation moisture.

\textbf{Humidity and Water Bodies:} The stable relative humidity (71.01 ±
16.97\%) reflects Lake Kastoria's moderating influence on local microclimate,
demonstrating the importance of water bodies in regional climate regulation.

This multi-sensor approach enables comprehensive environmental monitoring that
captures both immediate spectral responses and underlying meteorological
drivers, essential for understanding ecosystem dynamics in Mediterranean
lake environments.

\subsubsection{Seasonal Analysis}

\begin{figure}[H]
    \centering
    \includegraphics[width=\textwidth]{figures/kastoria_seasonal_analysis.png}
    \caption{Seasonal analysis of meteorological variables showing box plots by season. Temperature exhibits typical Mediterranean seasonality (Spring: 9.4°C, Summer: 22.0°C, Autumn: 13.2°C, Winter: 2.0°C). Precipitation shows winter maximum, humidity increases in winter (86.5\%), and wind speed remains relatively consistent across seasons. Numbers indicate seasonal means.}
    \label{fig:seasonal}
\end{figure}

\subsection{Objective 4: Interactive Web Applications}

\subsubsection{Web Mapping Architecture}

Three interactive web applications were developed, each serving different user
needs:

\begin{enumerate}
    \item \textbf{Comprehensive Map:} Full-featured application with all data sources
    \item \textbf{Advanced Application:} Enhanced functionality using Leafmap
    \item \textbf{Analysis Dashboard:} Results-focused interface with embedded analytics
\end{enumerate}

\subsubsection{Web Services Integration Challenges and Solutions}

The integration of OGC web services from geodata.gov.gr presented several
technical challenges that required adaptive solutions:

\textbf{WFS Connectivity Issues:} Initial attempts to access Greek geodata
portal WFS services encountered connectivity limitations. The system
implemented fallback mechanisms, creating alternative administrative boundaries
when live services were unavailable.

\textbf{CRS Compatibility:} Ensuring proper coordinate reference system
alignment between the study area polygon (EPSG:4326), raster data, and WMS
services required careful transformation and validation.

\textbf{Service Performance:} WMS layer integration was optimized for web
performance while maintaining cartographic quality, balancing data richness
with interactive responsiveness.

These challenges highlight the importance of robust error handling and
alternative data strategies when developing operational geospatial web
applications that depend on external services.

\subsubsection{Technical Implementation}

The web applications integrate multiple technologies and data sources:

\begin{lstlisting}[caption=WMS service integration example]
wms_services = {
    'geodata_gov_gr': {
        'url': 'https://geodata.gov.gr/geoserver/ows',
        'layers': {
            'administrative': 'geodata:administrative_boundaries_kallikratis',
            'roads': 'geodata:road_network_l',
            'settlements': 'geodata:settlements_p'
        }
    }
}
\end{lstlisting}

\subsubsection{Interactive Features}

The web applications provide comprehensive interactive functionality:

\begin{itemize}
    \item \textbf{Multi-layer visualization:} Toggle between satellite imagery, administrative boundaries, and thematic layers
    \item \textbf{Interactive popups:} Embedded charts showing spectral indices and meteorological time series
    \item \textbf{Measurement tools:} Distance and area calculation capabilities
    \item \textbf{Drawing tools:} User annotation and markup functionality
    \item \textbf{Multiple basemaps:} OpenStreetMap, satellite imagery, CartoDB themes
    \item \textbf{WMS integration:} Real-time access to Greek national geodata services
\end{itemize}

\begin{figure}[H]
    \centering
    \includegraphics[width=\textwidth]{figures/kastoria_web_application_screenshot.png}
    \caption{Screenshot of the interactive web mapping application showing successful integration of all data sources. The interface displays the study area polygon (89.05 km²) with interactive popup containing area, geometry, and coordinates information. Multiple markers show meteorological station (blue), spectral analysis (green), and time series (orange) data integration points.}
    \label{fig:web_app}
\end{figure}

\section{Conclusion}

This study successfully demonstrates a comprehensive geospatial big data
analysis workflow for the Kastoria study area. The integration of
multi-temporal Sentinel-2 imagery, meteorological observations, and
administrative vector data provides a robust foundation for environmental
monitoring and analysis.

\begin{thebibliography}{99}

    \bibitem{sentinel2}
    European Space Agency. (2023). \textit{Sentinel-2 User Handbook}. ESA Standard Document.

    \bibitem{nasa_power}
    NASA POWER Project. (2023). \textit{Prediction of Worldwide Energy Resources}. NASA Langley Research Center. \url{https://power.larc.nasa.gov/}

    \bibitem{rasterio}
    Gillies, S. (2023). \textit{Rasterio: Access to Geospatial Raster Data}. Python Package. \url{https://rasterio.readthedocs.io/}

    \bibitem{geopandas}
    Jordahl, K., Van den Bossche, J., Fleischmann, M., et al. (2023). \textit{GeoPandas: Python Tools for Geographic Data}. \url{https://geopandas.org/}

    \bibitem{folium}
    Python Visualization Development Team. (2023). \textit{Folium: Python Data, Leaflet.js Maps}. \url{https://python-visualization.github.io/folium/}

    \bibitem{ogc_standards}
    Open Geospatial Consortium. (2023). \textit{OGC Web Map Service (WMS) Implementation Standard}. OGC Document 06-042r7.

    \bibitem{ndvi_theory}
    Tucker, C. J. (1979). Red and photographic infrared linear combinations for monitoring vegetation. \textit{Remote Sensing of Environment}, 8(2), 127-150.

    \bibitem{ndwi_theory}
    McFeeters, S. K. (1996). The use of the Normalized Difference Water Index (NDWI) in the delineation of open water features. \textit{International Journal of Remote Sensing}, 17(7), 1425-1432.

    \bibitem{time_series_analysis}
    Verbesselt, J., Hyndman, R., Newnham, G., \& Culvenor, D. (2010). Detecting trend and seasonal changes in satellite image time series. \textit{Remote Sensing of Environment}, 114(1), 106-115.

    \bibitem{web_gis}
    Kraak, M. J. (2004). The role of the map in a Web-GIS environment. \textit{Journal of Geographical Systems}, 6(2), 83-93.

\end{thebibliography}

\newpage
\appendix

\section{Technical Specifications}

\subsection{Software Environment}

\begin{table}[H]
    \centering
    \caption{Software Dependencies and Versions}
    \begin{tabular}{@{}ll@{}}
        \toprule
        Component  & Version \\
        \midrule
        Python     & 3.x     \\
        rasterio   & 1.3+    \\
        geopandas  & 0.14+   \\
        pandas     & 2.0+    \\
        matplotlib & 3.7+    \\
        folium     & 0.14+   \\
        numpy      & 1.24+   \\
        shapely    & 2.0+    \\
        requests   & 2.31+   \\
        \bottomrule
    \end{tabular}
\end{table}

\subsection{Data File Summary}

\begin{table}[H]
    \centering
    \caption{Generated Data Files}
    \begin{tabular}{@{}lll@{}}
        \toprule
        Filename                                    & Type    & Description                 \\
        \midrule
        kastoria\_study\_area.geojson               & Vector  & Study area polygon          \\
        kastoria\_meteorological\_data.csv          & Tabular & NASA POWER time series      \\
        kastoria\_spectral\_indices\_timeseries.csv & Tabular & Calculated indices          \\
        kastoria\_comprehensive\_map.html           & Web     & Interactive map application \\
        kastoria\_dashboard.html                    & Web     & Analysis dashboard          \\
        \bottomrule
    \end{tabular}
\end{table}

\subsection{Code Repository Structure}

\begin{verbatim}
kastoria_analysis/
|-- explore.py          # Objective 1 implementation
|-- analyze.py          # Objective 2 implementation  
|-- time_series_analysis.py   # Objective 3 implementation
|-- web_gis.py          # Objective 4 implementation
|-- data/
|   |-- Kastoria.tif
|   |-- Kast_RefData_26Classes.tif
|   +-- kastoria_study_area.geojson
|-- outputs/
|   |-- *.csv (time series data)
|   |-- *.html (web applications)
|   +-- *.png (static visualizations)
+-- README.md
\end{verbatim}

\section{Complete Code Listings}

The implementation is also available on GitHub at
\url{https://github.com/panos-span/multispectral_time_series}. Key algorithmic
components include:

\begin{enumerate}
    \item Polygon validation system with landmark checking
    \item Correct rasterio.mask implementation for spatial clipping
    \item Spectral indices calculation with proper error handling
    \item Time series analysis with trend detection
    \item Interactive web mapping with WMS integration
\end{enumerate}

\end{document}